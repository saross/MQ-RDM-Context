% Latex template: https://github.com/mqTeXUsers/Macquarie-University-Beamer-Theme

% Slide Masters:

% Title
% Text
% 2 column
% Full-image
% Bibliography
% Closing
 
\documentclass[aspectratio=169, 12pt]{beamer} % Aspect ratio
% https://tex.stackexchange.com/a/14339/5483 
% Possible values: 1610, 169, 149, 54, 43 and 32.
% 169 = 16:9

\PassOptionsToPackage{table}{xcolor}    %https://tex.stackexchange.com/a/5365/5483

\usetheme{macquarie}
\usepackage{multicol} % https://tex.stackexchange.com/a/396018/5483

\usepackage[english]{babel}       % Set language
% \usepackage[utf8x]{inputenc}      % Set encoding
\usepackage{colortbl}
\mode<presentation>           % Set options
{
  \usetheme{default}          % Set theme
  \usecolortheme{default}         % Set colors
  \usefonttheme{default}          % Set font theme
  \setbeamertemplate{caption}[numbered] % Set caption to be numbered
}

% Uncomment this to have the outline at the beginning of each section highlighted.
%\AtBeginSection[]
%{
%  \begin{frame}{Outline}
%    \tableofcontents[currentsection]
%  \end{frame}
%}

\usepackage{graphicx}         % For including figures
\usepackage{booktabs}         % For table rules
\usepackage{hyperref}         % For cross-referencing


\usepackage{enumitem} % https://tex.stackexchange.com/a/2292/5483

%https://tex.stackexchange.com/a/371844/5483
\setbeamerfont{bibliography entry author}{size=\tiny}
\setbeamerfont{bibliography entry title}{size=\tiny}
\setbeamerfont{bibliography entry location}{size=\tiny}
\setbeamerfont{bibliography entry note}{size=\tiny}
\setbeamerfont{bibliography item}{size=\tiny}

%https://tex.stackexchange.com/q/333587/5483
%TODO SHAWN REPLACE OSF URL
%\setbeamertemplate{footline}{\strut~\texttt{https://osf.io/v5jp7/}\hfill\insertframen%umber~/~\inserttotalframenumber\strut~~~}

\title{Research transparency and Data Management} % Presentation title
\author{Shawn A Ross}               % Presentation author
\institute{Office of the Deputy Vice-Chancellor (Research)}         % Author affiliation
\date{\today}                 % Today's date  
\begin{document}

% Title page
% This page includes the informations defined earlier including title, author/s, affiliation/s and the date
% \begin{frame}[noframenumbering]

\maketitle

  
% \end{frame}

% Outline
% This page includes the outline (Table of content) of the presentation. All sections and subsections will appear in the outline by default.
\begin{frame}{Strategies for field data capture infrastructure}
  \tableofcontents
\end{frame}

% The following is the most frequently used slide types in beamer
% The slide structure is as follows:
%
%\begin{frame}{<slide-title>}
% <content>
%\end{frame}

% Slides to speak to at CAA2019

\section{CAA2019 short presentation}

\begin{frame}{Perceptions of the reproducibility crisis}
  \begin{figure}[H]
    \centering
        \includegraphics[height=.7\textheight]{figures/reproducibility-graphic-online1.jpeg}
        \caption{Is there a reproducibility crisis? \cite{Baker2016-cf}}
        \label{fig:figure1a}
  \end{figure}
\end{frame}

\begin{frame}{Level 2 TOP Guidelines for authors (excerpt)}
  
    \begin{enumerate}[label=\arabic*.]
        \setcounter{enumi}{1}
        % This increments the enumerate counter by 1.
        
        \item Authors using original data must:
        \begin{enumerate}[label=\alph*.]

            \item make the data available at a trusted digital repository [...]
            \item include all variables, treatment conditions, and observations described in the manuscript.
            \item provide a full account of the procedures used to collect, preprocess, clean, or generate the data.
            \item provide program code, scripts, codebooks, and other documentation sufficient to precisely reproduce all published results.
            \item provide research materials and description of procedures necessary to conduct an independent replication of the research.
        \end{enumerate}
    \end{enumerate}
    \cite{Osf2014-pf}
\end{frame}

\begin{frame}{Scalable approaches to data and analysis}
  \begin{figure}[H]
    \centering
        \includegraphics[height=.7\textheight]{figures/Ocean-Health-Index.jpg}
        \caption{Better science in less time, illustrated by the Ocean Health Index project. \cite{Stewart_Lowndes2017-lj}}
        \label{fig:figure2a}
  \end{figure}
\end{frame}

\begin{frame}{`Small data' research}
 \begin{figure}[H]
    \centering
        \includegraphics[height=.75\textheight]{figures/Archaeologists-standards.png}
        \caption{Archaeologists contemplate data standards (FAIMS Stocktaking, 2012)}
        \label{fig:figure4a}
 \end{figure}
\end{frame}

\begin{frame}{The data lifecycle}
 \begin{figure}[H]
    \centering
        \includegraphics[height=.75\textheight]{figures/research-data-life-diagram.png}
        \caption{\cite{Jisc2018-gx} Image CC-BY-ND}
        \label{fig:figure5a}
 \end{figure}
\end{frame}

\begin{frame}{FAIMS Mobile software}
 \begin{figure}[H]
    \centering
        \includegraphics[height=.75\textheight]{figures/FAIMS-screenshots.png}
        \caption{FAIMS Mobile: GIS and `picture dictionaries'}
        \label{fig:figure8a}
 \end{figure}
\end{frame}

\begin{frame}{Field data capture infrastructure: key messages}
    \begin{itemize}[label=\textbullet]
        \item We deserve research-specific software.
        \item Diverse practices and limited resources require generalised software.
        \item Do one thing well with modular and federated software (but slice the pie thoughtfully).
        \item Open-source software has advantages (but is difficult to sustain). 
        \item Scope requirements carefully.
        \item Invest in outreach and engagement.
    \end{itemize}
\end{frame}


\begin{frame}{Generalised}
 \begin{figure}[H]
    \centering
        \includegraphics[height=.75\textheight]{figures/FAIMS-generalised.png}
        \caption{FAIMS Mobile customisations on GitHub}
        \label{fig:figure11a}
 \end{figure}
\end{frame}

\begin{frame}{Modular and federated}
 \begin{figure}[H]
    \centering
        \includegraphics[height=.75\textheight]{figures/FAIMS-federated.png}
        \caption{FAIMS Mobile federation}
        \label{fig:figure13a}
 \end{figure}
\end{frame}

\begin{frame}{Challenges and paths forward}
  How do we get from where we are now to where we want to be?
      \begin{itemize}[label=\textbullet]
        \item Understand the evolving expectations of transparent research. 
        \item Look past desktop software (Excel, ARCGIS, Filemaker, Access, etc.).
        \item Rally around emerging research- and domain-specific solutions (even if imperfect).
        \item Overcome `not invented here'; you don't need a bespoke solution.
        \item Budget for `ground-up' transparency (data and code). Up-front costs will be high but offer longer-term payoffs (in costs, time, and quality).
        \item Implement (and budget for) fundamental good practice in data and code management before other technologies.
        \item Improve research design (prioritise approach over methods) \cite{Muthukrishna2019-kt, Hole1973-cy}
    \end{itemize}
\end{frame}

% In-depth information for reference
\section{Transparency and reproducibility}

\begin{frame}{The `reproducibility crisis'}
  For nearly a decade the reproducibility crisis has featured in the scientific literature \cite{Jasny2011-bw, Baker2016-cf, Munafo2017-bj}. Low reproducibility rates have emerged from large-scale studies:
    \begin{itemize}[label=\textbullet]
        \item Results from only 39\% of psychology studies could be reproduced \cite{Open_Science_Collaboration2015-vf}.
        \item Even lower reproducibility rate in biomedical research \cite{Begley2012-xt,Prinz2011-za}.
    \end{itemize}
\end{frame}

\begin{frame}{Perceptions of the reproducibility crisis}
  \begin{figure}[H]
    \centering
        \includegraphics[height=.7\textheight]{figures/reproducibility-graphic-online1.jpeg}
        \caption{Is there a reproducibility crisis? \cite{Baker2016-cf}}
        \label{fig:figure3}
  \end{figure}
\end{frame}

\begin{frame}{The response: improved rigour and transparency}
  Key guidelines to good practice:
    \begin{itemize}[label=\textbullet]
        \item Findable, Accessible, Interoperable, and Reusable (FAIR) data \cite{Wilkinson2016-mr, Go-fair2017-vs}.
        \item Transparency and Openness Promotion (TOP) guidelines \cite{Nosek2015-wm}.
        \item Data transparency toolkit \cite{Perkel2018-rw}.
    \end{itemize}
\end{frame}

\begin{frame}{The response: from guidelines to mandates}
  Recent mandates for transparency or reproducibility:
    \begin{itemize}[label=\textbullet]
        \item Nature: Transparency Upgrade \cite{Nature2017-lq}.
        \item Nature: FAIR data in Earth science \cite{Nature2019-ng}.
        \item Copernicus: FAIR data in atmospheric sciences \cite{Van_Edig2018-bu}.
        \item Not just the natural sciences: AJPS requires data and code \cite{Jacoby2017-lw, Ajps2015-ex}.
        \item TOP Guidelines have 5000 signatories, including publishers representing 1000 journals \cite{Cos2019-mr}.
    \end{itemize}
\end{frame}

\begin{frame}{TOP Guidelines: publisher adoption}
  \begin{figure}[H]
    \centering
        \includegraphics[height=.7\textheight]{figures/TOP-landscape.png}
        \caption{The Landscape of Open Data Policies \cite{Mellor2018-bf}}
        \label{fig:figure2}
  \end{figure}
\end{frame}

% https://tex.stackexchange.com/a/2292/5483
% https://ctan.org/pkg/enumitem?lang=en

\begin{frame}{Level 2 TOP Guidelines for authors (excerpt)}
  
    \begin{enumerate}[label=\arabic*.]
        \setcounter{enumi}{1}
        % This increments the enumerate counter by 1.
        
        \item Authors using original data must:
        \begin{enumerate}[label=\alph*.]

            \item make the data available at a trusted digital repository [...]
            \item include all variables, treatment conditions, and observations described in the manuscript.
            \item provide a full account of the procedures used to collect, preprocess, clean, or generate the data.
            \item provide program code, scripts, codebooks, and other documentation sufficient to precisely reproduce all published results.
            \item provide research materials and description of procedures necessary to conduct an independent replication of the research.
        \end{enumerate}
    \end{enumerate}
    \cite{Osf2014-pf}
\end{frame}

\begin{frame}{What does this mean? Are we ready?}
  Emerging good practice - and publisher and funder policies - mean:
    \begin{itemize}[label=\textbullet]
        \item Comprehensive, FAIR datasets will be deposited in domain-specific repositories. Data, and especially metadata, quality will be higher.
        \item Data will be captured digitally as early in research as possible, and provenance / version history maintained.
        \item Research approach, processes, and procedures will be documented.
        \item Data processing and analysis will use code (not Excel or ARCGIS!) 
        \item Code will be documented and published for reuse.
        \item Further steps taken for analytical reproducibility (use of OSS, version control, automation, containerisation, etc.). 
    \end{itemize}
\end{frame}

\begin{frame}{Beyond compliance: large-scale research}
    The same approaches that facilitate transparency and reproducibility support the kind of scalable and synthetic research that can address archaeological `grand challenges'. \cite{Kintigh2014-ub}
        \begin{itemize}[label=\textbullet]
            \item Paper data capture and manual digitisation and cleaning don't scale.
            \item Email and desktop software don't scale.
    \end{itemize}
\end{frame}

\begin{frame}{Scalable approaches to data and analysis}
  \begin{figure}[H]
    \centering
        \includegraphics[height=.7\textheight]{figures/Ocean-Health-Index.jpg}
        \caption{Better science in less time, illustrated by the Ocean Health Index project. \cite{Stewart_Lowndes2017-lj}}
        \label{fig:figure14}
  \end{figure}
\end{frame}

\section{`Small data' infrastructure across the data lifecycle}

\begin{frame}{`Small data' research}
 \begin{figure}[H]
    \centering
        \includegraphics[height=.75\textheight]{figures/Archaeologists-standards.png}
        \caption{Archaeologists contemplate data standards (FAIMS Stocktaking, 2012)}
        \label{fig:figure7}
 \end{figure}
\end{frame}

\begin{frame}{Context: the challenge of `small data'}
    `Long tail' research: most field data is small data \cite{Borgman2015-rh}
    \begin{itemize}[label=\textbullet]
        \item Smaller scale; smaller communities; local control.
        \item Diverse questions, approaches, and methods.
        \item Heterogeneous data; variety of content, structure.
        \item Data and infrastructure emerge from fieldwork. 
        \item Relative lack of standards.
        \item Limited infrastructure and funding.
        \item Challenges associated with big(ger) data from photogrammetry, SfM, video, geophysics, etc., will exacerbate these problems.
    \end{itemize}
\end{frame}

\begin{frame}{The data lifecycle}
 \begin{figure}[H]
    \centering
        \includegraphics[height=.75\textheight]{figures/research-data-life-diagram.png}
        \caption{\cite{Jisc2018-gx} Image CC-BY-ND}
        \label{fig:figure9}
 \end{figure}
\end{frame}

\begin{frame}{Infrastructure across the data lifecycle}
    Consider the infrastructure needed to manage the three main phases of the data lifecycle
    \begin{itemize}[label=\textbullet]
        \item Publication (most mature): domain-specific repositories.
        \item Processing and analysis (less mature): project-level code \cite{Stewart_Lowndes2017-lj}, then Virtual Labs / Science Gateways, like \cite{Alveo2019-tk} in language analysis.
        \item Capture (least mature): most varied, needs to work offline under difficult conditions. Commercial solutions insufficient \cite{Bureau_of_Reclamation2017-xl}.
    \end{itemize}
\end{frame}


\section{From current practice to better practice}

\begin{frame}{Challenges and paths forward}
  How do we get from where we are now to where we want to be?
      \begin{itemize}[label=\textbullet]
        \item Understand the evolving expectations of transparent research. 
        \item Look past desktop software (Excel, ARCGIS, Filemaker, Access, etc.).
        \item Rally around emerging research- and domain-specific solutions (even if imperfect).
        \item Overcome `not invented here'; you don't need a bespoke solution.
        \item Budget for `ground-up' transparency (data and code). Up-front costs will be high but offer longer-term payoffs (in costs, time, and quality).
        \item Implement (and budget for) fundamental good practice in data and code management before other technologies.
        \item Improve research design (prioritise approach over methods) \cite{Muthukrishna2019-kt, Hole1973-cy}
    \end{itemize}
\end{frame}
% \bibliographystyle{apalike}

% Adding the option 'allowframebreaks' allows the contents of the slide to be expanded in more than one slide.
% The "1" comes from the outer theme"

\section{References}

\begin{multicols}{2}[]
\bibliography{references}
\bibliographystyle{apalike}
\end{multicols}


% \begin{frame}[allowframebreaks]{References}
  
%   \bibliography{references}
%   \bibliographystyle{apalike}
% \end{frame}


\begin{frame}{Thank you!}

This presentation is available at:
\texttt{https://osf.io/v5jp7/}

Source code for this presentation is available at: \texttt{https://github.com/saross/CAA-Ross-FAIMS}.

This work is licensed under a Creative Commons Attribution 4.0 International License.

\end{frame}



\end{document}
